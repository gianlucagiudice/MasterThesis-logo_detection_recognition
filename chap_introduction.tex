\chapter{Introduction}
\label{chap:introduction}

\section{Logo detection and recognition}

Logo detection and recognition is a Computer Vision (CV) task that consists of detecting logos in any type of image and classifying the detected logo, for example, by producing in output the brand name associated with that logo. 

The firsts studies in this field date back to the 1993 \cite{doermann1993logo}, yet this task is becoming increasingly important in a variety of application. Some examples of studies relative to this problems aimed to monitor the brand visibility on social media \cite{7492197}, protect the intellectual property protection (IPP) on e-commerce platforms \cite{jin2020open}, develop online video advertising system \cite{cheng2017video} and in recent years, the detection of logos using visual information can provide scene understanding in retail environments to help the development of autonomous checkout systems \cite{mata2022standardsim}.

There are several challenges in logo detection and classification, starting from the fact that a symbol composed by text and images can be considered a logo, but there is no formal definition of what a logo is. In fact, these can be created from text using a lot of different typographic styles, any particular graphic consisting of many colors, or even a combination of the two.
There are a huge variety of logos and this problem has both high intra-class and inter-class variations, since the same brand can have very different logos (e.g. only a stylized text version and an graphic
 version) and logos which belong to different brands might look very similar.

Since many new brands are constantly being created and each brand has its own logo, it is necessary to develop systems that keep up with the introduction of new logos. A method for logo detection and recognition should take into account this particular aspect of the problem and should correctly recognize each new logo.

For this reason, there is the need to develop a system which is able to adapt to these changes where standard closed-set classification techniques would fail. One possible approach to this problem could be to train a new classifier each time a new logo is introduced. Unfortunately, this method is very inefficient and unfeasible for large scale datasets of logos. Moreover, retraining a model in such a way would require to store a lot of examples, since both the data from the previous logos and the new ones would be need. To this reason, systems in which logo detection and recognition is performed in an open environment have been proposed in the literature \cite{fehervari2019scalable,li2022seetek}.

\section{Class incremental learning}
Incremental learning aims to develop artificially intelligent
systems that can continuously learn to address new tasks
from new data while preserving knowledge learned from
previously learned tasks \cite{masana2020class}.

Natural systems are intrinsically incremental
where new knowledge is continuously learned over time
while existing knowledge is preserved \cite{wu2019large}. 
The main problem in incremental learning is known as the stability-plasticity dilemma and it holds for both artificial and biological neural systems. The idea is that a learning system requires plasticity for the integration of new knowledge, but also stability in order to prevent the forgetting of previous knowledge \cite{mermillod2013stability}. Excessive plasticity would lead to forget all the previous knowledge (referred to as catastrophic forgetting \cite{grossberg2013adaptive}), whereas too much stability would prevent the ability to learn novel concepts. The main point is to achieve a trade-off between stability and plasticity. 

Many real-world applications require incremental learning capabilities: intelligent robots during their lifetime \cite{thrun1995lifelong}, face recognition system \cite{li2017incremental} and autonomous driving \cite{pierre2018incremental}. Logo detection and recognition is another example where class incremental learning is well suited to address the problem, considering the constant appearance of new logos discussed in the previous section. Using this technique it is possible to create a system which detects and recognizes an initial set of logos and, when necessary, enriches the acquired knowledge.

\section{Proposed approach}

\begin{figure}
    \begin{center}
        \includegraphics[width=\columnwidth]{images/pipeline.drawio.png}
    \end{center}
    \caption{Simplified system pipeline.}
    \label{fig:system-pipeline}
\end{figure}


The goal of this thesis is to develop a system that can detect and recognize logos in an image. An important factor that will be considered is the ability to recognize those logos that were not available before training, and this will be possible by taking advantage of state of the art incremental learning techniques.

The proposed system is conceptually simple, it consists of two deep learning models and follows a general pipeline for the task \cite{bianco2017deep}. The two main steps, shown in \autoref{fig:system-pipeline}, consist in:

\begin{enumerate}
    \item \textbf{Object Proposal} for logo detection
    \item \textbf{Classification} for logo recognition
\end{enumerate}

The first model is an class-agnostic logo detector based on YOLOv5 \cite{glenn_jocher_2021_5563715}. Given an input image, the purpose of this first step is to produce as many cropped portions of the image as there are logos in the image. These cropped regions are called region proposals and correspond to what the model considers to be logos.

The next step exploits the regions of interest generated by the detector and proceeds with the actual recognition of logos. This step is purely a classification task, and it is where the problem of recognizing new logos not present in the initial dataset used to train the model arises. For this reason, it is in this step that there is a need to create a model using incremental learning techniques; by doing so, it will be possible to recognize new logos while maintaining the knowledge of old ones.

We say that, unlike the second step, there is no need to develop a detector with incremental learning techniques. This is justified by the fact that the general idea of what a generic logo is can be learned and well-approximated using only an initial set of logos. The subsequent introduction of new logos will not disrupt the general idea of what a logo is, therefore there is no need to change the knowledge initially learned.

\vspace{1.5\baselineskip}
This thesis is structured as follows: the second chapter describes the state of the art regarding object detection, logo recognition and class incremental learning algorithms; the third chapter describes the dataset used for tackle this problem; the fourth chapter is a detailed description of the development and the techniques adopted and in following chapter will be discussed the experiments and results obtained using the developed system. The last chapter is relative to the conclusions of this work and some considerations about future works.